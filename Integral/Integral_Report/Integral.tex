\documentclass[a4paper,12pt]{report}

\usepackage[utf8]{inputenc}
\usepackage[T1, T2A]{fontenc}
\usepackage[english, russian]{babel}
\usepackage[final]{graphicx} 
\usepackage{geometry} 
\usepackage{indentfirst}
\usepackage{amsmath}
\usepackage{pdfpages}
\usepackage{nccmath}

\geometry{left=2cm}
\geometry{right=1.5cm}
\geometry{top=1cm}
\geometry{bottom=2cm}

\begin{document}

\begin{titlepage}
    \begin{center}
        \textbf{МОСКОВСКИЙ ГОСУДАРСТВЕННЫЙ УНИВЕРСИТЕТ}\\
        имени М.В. Ломоносова\\
        \vspace{0.5cm}
        \textbf{Факультет вычислительной математики и кибернетики}\\
        \vspace{5.5cm}
        \textbf{\Large Компьютерный практикум по учебному курсу}\\
        \textbf{\Large «ВВЕДЕНИЕ В ЧИСЛЕННЫЕ МЕТОДЫ»}\\
        \vspace{4.5cm}
        \textbf{\Large ЗАДАНИЕ № 1}\\
        \vspace{1cm}
        \textbf{\LARGE ОТЧЕТ}\\
        \textbf{\LARGE о выполненном задании}\\
        \vspace{2cm}
        студента 214 учебной группы факультета ВМК МГУ\\
        \textbf{\large Исмурзенова Абая Ерденовича}\\
        \vspace{\fill}
        гор. Москва\\
        2024 г.
    \end{center}
\end{titlepage}
	
	%содержание
	\newpage
	\tableofcontents
	
	%текст
	
	\newpage
	\chapter{}
	\section{Постановка задачи}
	
    Найдите приближенное значение интеграла методом трапеций, разбив интервал интегрирования на $n$ равных частей, где $n = 16, 32, 64$.\\[0.5cm]
    	
    	
    	Интеграл:\\[-1.5cm]

	\begin{equation*}
		\hspace{-7cm}
    		\int_{a}^{b} \frac{1}{\left(25x^2 + 1\right)\sqrt{3x - x^2}} \, dx
	\end{equation*}
	
	Рассмотрите два отрезка интегрирования:
	
	\begin{enumerate}
		\item a = 1, b = 2
		\item a = 0, b = 3
	\end{enumerate}
	
	Сравните результаты с аналитическим значением интеграла.
	Подберите более эффективный численный метод вычисления интеграла для второй задачи
	
	\section{Цели и задачи}
	
	\begin{enumerate}
		\item Найти приблеженние значение данного интеграла методом трапеций 
		\item Подобрать более эффективный метод вычисления интеграла для второй подзадачи
		
	\end{enumerate}
	
	\section{Описание метода трапеций}
	Нам нужно приблеженно вычислить определенный интеграл
	$\int_{a}^{b} f(x) dx$, подынтегральная функция которого $y = f(x)$ непрерывна на отрезке $[a, b]$. Для этого разделим отрезок $[a, b]$ на несколько равных интервалов длины h точками $a = x_0 < x_2 < ... < x_{n-1} < x_n = b$. Количество полученных интервалов обозначим за n.
	
	
\end{document}
